\section{Постановка задачи доверительного оценивания, доверительный интервал.}
%:: NOTE a:dzhiblavi
%:: NOTE check

При оценивании параметров или характеристик распределений мы в качестве
результата получаем числовое значение $\hat{g}(X) \in g(\Theta)$. Такой способ 
оценивания мы называем \textit{точечной оценкой}. Заранее не понятно, 
насколько результат соответствует действительности. Для того, чтобы можно 
было оценивать качество результата, нужно предъявлять не точку, а подмножество
в $g(\Theta)$, содержащее в некотором смысле наиболее подходящие значения.

Задача доверительного оценивания ставится следующим образом: задана величина
$\gamma \in (0, 1)$, называемая \textit{уровнем надежности}. По заданному
наблюдению $X$ и значению надежности требуется построить доверительную область
надежности.

\begin{definition}
    \textit{Доверительной областью надежности} называется $\widetilde{G}_\gamma
    \subseteq G = g(\Theta)$, обладающая свойством:
    \[
        \forall \theta \in \Theta~ P_\theta(g(\theta) \in \widetilde{G}_\gamma) 
        \geqslant \gamma
    .\]
    То есть множество, с достаточной вероятностью содержащее оцениваемую 
    величину.
\end{definition}

\begin{definition}
    В случае одномерной оценки чаще всего доверительные области надежности
    выбирают в виде промежутков, которые называются \textit{доверительными
    интервалами}.
\end{definition}

\begin{definition}
    В асимптотическом случае (когда имеется последовательность оценок и
    статистических экспериментов) последовательность \textit{асимптотических
    областей надежности} $\widetilde{G}_{n, \gamma}$ задается условием:
    \[
        \forall \theta \in \Theta~ \lim{P_{n, \theta}(g(\theta) \in 
        \widetilde{G}_{n, \gamma})} \geqslant \gamma
    .\]
\end{definition}

\begin{definition}
    Аналогично задается последовательность асимптотических доверительных
    интервалов в случае одномерной характеристики.
\end{definition}

