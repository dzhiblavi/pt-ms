\section{Выборочный метод построения оценок одномерных характеристик. Асимптотическая нормальность оценки. Построение асимптотического доверительного интервала на базе асимптотической нормальности.}
%:: NOTE a:zagretdinov
%:: NOTE check

Есть три метода построения оценок: \\

\begin{enumerate}
  \item Выборочный метод
  \item Метод моментов
  \item Метод максимального правдоподобия
\end{enumerate}


\subsection{Описание выборочного метода}

Этот метод основывается на знании того, что ЭФР - $F_n(t)$ - является "хорошей" оценкой функции распределения $F_x(t)$.

ЭФР - $F_n(t)$ является функцией распределения дискретной случайной величины $Y$, имеющей следующий ряд распределения:

%:: NOTE запятую поставить

\begin{center}
\begin{tabular}{ c|c|c|c|c }
  $Y_i$ & $x_{(1)}$ & $x_{(2)}$ & $...$ & $x_{(n)}$ \\
 \hline
 $p_i$ & $1/n$ & $1/n$ & $...$ & $1/n$ \\
\end{tabular}
\end{center}

где $x_{(1)}, ..., x_{(n)}$ упорядоченная выборка:
\[
  x_{(1)} \le x_{(2)} \le ... \le x_{(n)}
.\]
