\section{Выборочный метод построения оценок одномерных характеристик.
    Асимптотическая нормальность оценки. Построение асимптотического доверительного
интервала на базе асимптотической нормальности.}
%:: NOTE a:zagretdinov
%:: NOTE check
%:: NOTE column 80
\subsection{Описание выборочного метода}
Этот метод основывается на знании того, что ЭФР $F_n(t)$ является ``хорошей''
оценкой функции распределения $F_x(t)$.

ЭФР $F_n(t)$ является функцией распределения дискретной случайной величины
$Y$, имеющей следующий ряд распределения:

%:: NOTE запятую поставить

\begin{center}
    \begin{tabular}{ c|c|c|c|c }
        $Y_i$ & $x_{(1)}$ & $x_{(2)}$ & $...$ & $x_{(n)}$ \\
        \hline
        $p_i$ & $1/n$ & $1/n$ & $...$ & $1/n$ \\
    \end{tabular}
\end{center}

где $x_{(1)}, ..., x_{(n)}$ упорядоченная выборка:
\[
    x_{(1)} \leqslant x_{(2)} \leqslant ... \leqslant x_{(n)}
.\]

В основе выборочного метода лежит идея:
любую характеристику генеральной совокупности $X$ оценивать при помощи соответствующей характеристики случайной величины $Y$. Естественно полученные таким образом оценки нужно изучать, проверять их свойства. С точки зрения квадратического риска они не всегда являются лучшими в соответствующем классе распределений.

\subsection{Асимптотическая нормальность}

\begin{definition}
    Последовательность оценок $\hat{g}_n$ характеристики $g(\theta)$ называется
    \textit{асимптотически нормальной с асимптотической дисперсией}
    $\Delta^2(\theta)
    \textgreater 0$, если случайная величина $Y_n = \sqrt{n}(\hat{g}_n -
    g(\theta))$ сходится по $P_{n, \theta}$ - распределению к нормальной
    случайной величине $Y$ с нулевым средним и дисперсией $\Delta^2(\theta)$:
    \begin{equation}
        Y_n \xrightarrow[n \to \infty]{P_{n, \theta}} Y \sim
        N(0, \Delta^2(\theta))\label{eq:1}
    \end{equation}
\end{definition}

Перепишем $\ref{eq:1}$
\[
    \hat{g}_n = g(\theta) + \frac{Y_n}{\sqrt{n}}, где Y_n \xrightarrow[n \to \infty]{P_{n, \theta}} Y \sim N(0, \Delta^2(\theta))
,\] то есть
$\hat{g}_n - g(\theta)$ — отклонение оценки от неизвестного значения
оцениваемой характеристики имеет приближенно нормальное распределение
с нулевым средним и дисперсией $\Delta^2(\theta)/n$.

\begin{definition}
    Величина $\Delta(\theta) > 0$ называется \textit{нормирующим множителем}.
\end{definition}

\subsection{Построение асимптотического доверительного интервала}
  Из асимптотической нормальности имеем
    \[
        P_{n, \theta}\parens*{\frac{Y_n}{\Delta(\theta)} < t}
        \xrightarrow[n \to +\infty]{P_{n, \theta}} F_{N(0, 1)}(t)
    .\]
    Отсюда оценим вероятность:
    \[
        P_{n, \theta}\parens*{|\hat{g}_n - g(\theta)| <
        T_{\frac{1 + \g}{2}}\frac{\Delta(\theta)}{\sqrt{n}}} \xrightarrow[n \to \infty]{} 2 \Phi(T_{\frac{1 + \g}{2}}) - 1 = \gamma
    .\]
    Здесь $\Phi$ -- функция распределения стандартного нормального закона, $\g$ -- надежность, $T_{\frac{1 + \g}{2}}$ -- $(1 + \g)/2$ - квантиль $N(0, 1)$. Пусть
    \[
        \delta_n = T_{\frac{1 + \g}{2}} \frac{\Delta(\theta)}{\sqrt{n}}
    .\]
    Получается, что
    \[
        (\hat{g}_n - \delta_n, \hat{g}_n + \delta_n)
    \]
    есть асимтотический доверительный интервал надежности $\g$.

\subsection{Свойства асимптотической нормальности оценок}

\begin{proposition}
    Для асимптотически нормальной оценки $\hat{g}_n$, при условии следующей сходимости моментов второго порядка:
    \begin{equation}
      E_{n, \theta}(Y_n^2) \xrightarrow[n \to \infty]{} E(Y^2) =
      \Delta^2(\theta)\label{eq:2},
    \end{equation}
    выполнено:
    \begin{itemize}
        \item $\hat{g}_n$ -- $\sqrt{n}$-несмещенная оценка.
        \item $nD_{n, \theta}(\hat{g}_n) = D_{n, \theta}(Y_n)
            \xrightarrow[n \to \infty]{} \Delta^2(\theta)$
        \item $nR_2(\hat{g}_n, \theta) = E_{n, \theta}(Y^2_n)
            \xrightarrow[n \to \infty]{} \Delta^2(\theta)$
    \end{itemize}
\end{proposition}
\begin{proof}
    \enewline
      Пользуясь сходимостью моментов второго порядка $\ref{eq:2}$, имеем:
    \[
       E_{n, \theta}(Y_n) \xrightarrow[]{} E(Y) = 0
    \]
    \[
       E_{n, \theta}(Y_n) = \sqrt{n}(E_{n, \theta}\hat{g}_n - g(\theta)) =
       \sqrt{n}b_{n, \theta}(\hat{g}_n) \xrightarrow[n \to \infty]{} 0
     % \end{align*}
    ,\]
    то есть смещение стремится к нулю быстрее, чем $1/\sqrt{n}$, получается, что $\hat{g}_n$ -- $\sqrt{n}$-несмещенная оценка.
    \[
        D_{n, \theta}(Y_n) = E_{n, \theta}(Y_n^2) - (E_{n, \theta}(Y_n))^2
        \xrightarrow[n \to \infty]{} \Delta^2(\theta)
    .\]
    \[
        D_{n, \theta}(\sqrt{n}(\hat{g}_n - g(\theta))) = nD_{n, \theta}
        (\hat{g}_n)
    \] пользуемся инвариантностью дисперсии относительно сдвига на $g(\theta)$ и выносим $\sqrt{n}$ из-под знака
    дисперсии.
    \[
        nD_{n, \theta}(\hat{g}_n) \xrightarrow[]{} \Delta^2(\theta)
    \]
    \[
        \sqrt{n}\sigma_{n, \theta}(\hat{g}_n) \xrightarrow[]{} \Delta(\theta)
    ,\]
    получили, что среднеквадратическое отклонение имеет порядок
    $\Delta(\theta)/\sqrt{n}$.
    Более того,
    \[
        nR_2(\hat{g}_n, \theta) = n(b_{n, \theta}^2(\hat{g}_n) +
        D_{n, \theta}(\hat{g}_n)) \xrightarrow[]{} 0 + \Delta^2(\theta)
    ,\]
    то есть для асимптотически нормальных оценок дисперсия и квадратический
    риск в асимптотике совпадают и равны $\Delta^2(\theta)/n$.
\end{proof}

%:: NOTE check me
\begin{proposition}
    Оценки $S^2_n$ и $\sigma^2_n$ - асимптотически нормальны с
    $\Delta^2(X) = E(X - EX)^4 - D^2(X)$ (при условии существования четвертых
    центральных моментов):
    \[
      \frac{\sigma^2_n - D(X)}{\Delta(X)}\sqrt{n}
      \xrightarrow[n \to \infty]{P_{n, \theta}} Y \sim N(0, 1)
    ,\] а точность выборочной дисперсии оценивается моментами 4-го порядка.
\end{proposition}
