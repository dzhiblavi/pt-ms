\section{Выборочный метод построения оценок одномерных характеристик.
Асимптотическая нормальность оценки. Построение асимптотического доверительного 
интервала на базе асимптотической нормальности.}
%:: NOTE a:zagretdinov
%:: NOTE check
%:: NOTE column 80

%:: NOTE не понял, зачем новый параграф
Есть три метода построения оценок:
\begin{enumerate}
  \item Выборочный метод
  \item Метод моментов
  \item Метод максимального правдоподобия
\end{enumerate}


\subsection{Описание выборочного метода}

%:: NOTE кавычки (я исправил)
%:: NOTE тире (тут они не нужны, но если и нужны, то они делаются так: --)
Этот метод основывается на знании того, что ЭФР $F_n(t)$ является ``хорошей''
оценкой функции распределения $F_x(t)$.

ЭФР $F_n(t)$ является функцией распределения дискретной случайной величины 
$Y$, имеющей следующий ряд распределения:

%:: NOTE запятую поставить

\begin{center}
\begin{tabular}{ c|c|c|c|c }
  $Y_i$ & $x_{(1)}$ & $x_{(2)}$ & $...$ & $x_{(n)}$ \\
 \hline
 $p_i$ & $1/n$ & $1/n$ & $...$ & $1/n$ \\
\end{tabular}
\end{center}

%:: NOTE \le -> \leqslant
где $x_{(1)}, ..., x_{(n)}$ упорядоченная выборка:
\[
  x_{(1)} \leqslant x_{(2)} \leqslant ... \leqslant x_{(n)}
.\]
