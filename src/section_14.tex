\section{Выборочный метод построения оценок одномерных характеристик.
Асимптотическая нормальность оценки. Построение асимптотического доверительного
интервала на базе асимптотической нормальности.}
%:: NOTE a:zagretdinov
%:: NOTE check
%:: NOTE column 80

%:: NOTE не понял, зачем новый параграф

%:: NOTE a:dolzhanskii
%:: NOTE Нужен более нейтральный стиль
Есть три метода построения оценок:
\begin{enumerate}
  \item Выборочный метод
  \item Метод моментов
  \item Метод максимального правдоподобия
\end{enumerate}


\subsection{Описание выборочного метода}

%:: NOTE кавычки (я исправил)
%:: NOTE тире (тут они не нужны, но если и нужны, то они делаются так: --)
Этот метод основывается на знании того, что ЭФР $F_n(t)$ является ``хорошей''
оценкой функции распределения $F_x(t)$.

ЭФР $F_n(t)$ является функцией распределения дискретной случайной величины
$Y$, имеющей следующий ряд распределения:

%:: NOTE запятую поставить

\begin{center}
\begin{tabular}{ c|c|c|c|c }
  $Y_i$ & $x_{(1)}$ & $x_{(2)}$ & $...$ & $x_{(n)}$ \\
 \hline
 $p_i$ & $1/n$ & $1/n$ & $...$ & $1/n$ \\
\end{tabular}
\end{center}

%:: NOTE \le -> \leqslant
где $x_{(1)}, ..., x_{(n)}$ упорядоченная выборка:
\[
  x_{(1)} \leqslant x_{(2)} \leqslant ... \leqslant x_{(n)}
.\]

%:: NOTE a:dolzhanskii
%:: NOTE Нужен более нейтральный стиль

В основе выборочного метода лежит идея:
любую характеристику генеральной совокупности $X$ оценивать при помощи соответствующей характеристики случайной величины $Y$. Естественно полученные таким образом оценки нужно изучать, проверить их свойства. С точки зрения квадратического риска они не всегда являются лучшими в соответствующем классе распределений.

\subsection{Асимптотическая нормальность, свойства асимптотической нормальности оценок}

Определение (для одномерного параметра $\theta$, $g(\theta)$)
Последовательность оценок $\hat{g}_n$ характеристики $g(\theta)$ $def$ асимптотически нормальной с ас. дисперсией $\Delta^2(\theta) \textgreater 0$, если сл. в. $Y_n = \sqrt{n}(\hat{g}_n - g(\theta))$ сходится по $P_{n, x}$ - распределению к нормальной сл. в. $Y$ с нулевым средним и дисперсией $\Delta^2(\theta)$
\[
  (1)~ Y_n \xrightarrow[n \to \infty]{P_{n, x}} Y \sim N(0, \Delta^2(p))
.\]

Перепишем $(1)$
\[
  \hat{g}_n = g(\theta) + \frac{Y_n}{\sqrt{n}}, где Y_n \xrightarrow[n \to \infty]{P_{n, x}} Y \sim N(0, \Delta^2(p))
.\], то есть
$\hat{g}_n - g(\theta)$ - отклонение оценки от неизвестного значения оцениваемой характеристики имеет приближенно нормальное распределение с нулевым средним и дисперсией $\frac{\Delta^2(\theta)}{n}$

%:: NOTE T letter

$\Delta(\theta)$ - $def$ нормирующим множителем и $P_{n, \theta}(\frac{Y_n}{\Delta(\theta)} \textless t) \xrightarrow[n \to \infty]{} F_{N(0, 1)}(t) \Rightarrow{} P_{n, x}(|\hat{g}_n - g(\theta)| \textless \frac{T(\Delta(\theta))} {\sqrt{n}}) \xrightarrow[n \to \infty]{} 2\Phi(T) - 1 = \g \Rightarrow{} T = \frac{1 + \g}{2}$ - квантиль $N(0, 1)$;
$\g$ - надежность, $\delta_n = T_{\frac{1 + \g}{2}}\frac{\Delta}{(\theta)}$ - точность оценки.
$(\hat{g}_n - \delta_n, \hat{g}_n + \delta_n)$ - асимптотически доверительный интервал надежности $\g$.
Если при этом $E_{n, x}(Y_n^2) \xrightarrow[n \to \infty] E(Y^2) = \Delta^2(\theta) (тогда E_{n, x}(Y_n) \xrightarrow[]{} E(Y) = 0)$, то $E_{n, x}(Y_n) = \sqrt{n}(E_{n, x}\hat(g)_n - g(\theta))
%:: NOTE b_{n, x}? letter
 = \sqrt{n}b_{n, x}(\hat(g)_n) \xrightarrow[n \to \infty]{} 0$
 т.е. смещение стремится к нулю быстрее, чем $\frac{1}{\sqrt{n}}$
- $\sqrt{n}$ несмещ $\hat{g}_n$
%:: NOTE check me

\[
  D_{n, x}(Y_n) = E_{n, x}(Y_n^2) - (E_{n, x}(Y_n))^2 \xrightarrow[n \to \infty]{} \Delta^2(\theta)
.\]

%:: NOTE check me
\[
  D_{n, x}(\sqrt{n}(\hat{g}_n - g(\theta))) = nD_{n, x}(\hat{g}_n)
.\] - инвариантна относительно сдвига и вынесение $\sqrt{n}$ из-под знака дисперсии

т.е $nD_{n, x}(\hat{g}_n) \xrightarrow[]{} \Delta^2(\theta)$
$\sqrt{n}\sigma_{n, x}(\hat{g}_n) \xrightarrow[]{} \Delta(\theta)$, средне квадратическое отклонение имеет порядок $\frac{\Delta(\theta)}{\sqrt{n}}$

\[
  nR_2(\hat{g}_n, \theta) = n(b_{n, x}^2(\hat{g}_n) + D_{n, x}(\hat{g}_n)) \xrightarrow[]{} 0 + \Delta^2(\theta)
.\]
т.е для асимптотически нормальных оценок дисперсия и квадратический риск в асимптотике совпадают.

%:: NOTE check me
3. Без док-ва. \textit{утверждение} Оценки $S^2_n$ и $\sigma^2_n$ - асимптотически нормальны с $Delta^2(X) = E(X - EX)^4 - D^2(X)$, если существуют четвертые центральные моменты.
т.е. $\frac{\sigma^2_n - D(X)}{\Delta(X)}\sqrt{n} \xrightarrow[n \to \infty]{P_{n, x}} Y \sim N(0, 1)$ и точность выборочной дисперсии оценивается моментами 4-го порядка.
