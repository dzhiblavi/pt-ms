\section{Свойства ЭФР в целом. Расстояние Колмогорова, Смирнова. Теоремы
Гливенко-Кантелли, Колмогорова, Мизеса-Смирнова. Построение доверительной
полосы для функции распределения.}
%:: NOTE a:dolzhanskii
%:: NOTE check

\textit{Со свойствами ЭФР можно ознакомиться в предыдущем разделе.}

\begin{definition}
  \textit{Расстояние Колмогорова}:
  \[
    \rho_{\infty} (F_n, F_x) = \sup_{t}|F_n(t) - F_x(t)|
  .\]
\end{definition}

\begin{definition}
  \textit{Расстояние Смирнова}:
  \[
    \rho^{2}_{2} (F_n, F_x) = \int\limits_{\mathbb{R}}(F_n(t) - F_x(t))^2
    dF_x(t)
  .\]
\end{definition}

\begin{theorem}(Гливенко-Кантелли)

  Пусть $\cF$ -- множество функций распределения.
  Тогда $\forall F_x(t) \in \cF$ верно:
  \[
    \rho_{\infty} (F_n, F_x) \xlongrightarrow[n \to \infty]{p=1} 0 \Rightarrow
    \rho_{\infty} (F_n, F_x) \xlongrightarrow[p]{} 0
  .\]

\end{theorem}

\begin{remark}
  $F_n(t)$ -- состоятельная оценка $F_x(t)$ в расстояниях Колмогорова и
  Смирнова.
\end{remark}

\begin{theorem}(Колмогоров)

  Пусть $\cF_c$ == множество всех непрерывных функций распределения.
  Тогда:
  \[
    P_x(\sqrt{n} \rho_{\infty} (F_n, F_x) < u) \xlongrightarrow[n \to \infty]{}
    \mathcal{K}(u) = \begin{cases}
      0,~ u = 0\\
      \sum\limits_{j = -\infty}^{+\infty} (-1)^j e^{-2 (ju)^2},~ u > 0
    \end{cases}
  .\]

\end{theorem}

\begin{remark}
  Используя теорему Колмогорова, можно построить доверительную полосу для
  функции распределения.
\end{remark}

\begin{definition}
  \textit{Доверительной полосой} называют часть плоскости, в которую с
  надежностью $\gamma$ попадает функция распределения генеральной совокупности:
  \[
  \text{полоса}
  \begin{cases}
    F^{-}_n(t) = max(0, F_n(t) - \frac{u_{\gamma}}{\sqrt{n}})\\
    F^{+}_n(t) = min(1, F_n(t) + \frac{u_{\gamma}}{\sqrt{n}})
  \end{cases}
  ,~ \text{где} \mathcal{K}(u_{\gamma}) = \gamma
  .\]
\end{definition}

\pagebreak

\begin{proposition}
  \[
    P_x(F^{-}_n(t) \leqslant F_x(t) \leqslant F^{+}_n(t))
    \xlongrightarrow[n \to \infty]{} \gamma
  .\]
\end{proposition}
\begin{proof}
  $0 \leqslant F_x(t) \leqslant 1$ всегда, тогда:
  \begin{align*}
    P_x(F^{-}_n(t) \leqslant F_x(t) \leqslant F^{+}_n(t)) = \\
    &=P_x(F_n(t) - \frac{u_{\gamma}}{\sqrt{n}} \leqslant F_x(t) \leqslant
    F_n(t) + \frac{u_{\gamma}}{\sqrt{n}}) \overset{\forall t}{\underset{}{=}} \\
    &\overset{\forall t}{\underset{}{=}}
    P_x(\sqrt{n} |F_x(t) - F_n(t)| \leqslant u_{\gamma})
    \overset{\forall t}{\underset{}{=}} \\
    &\overset{\forall t}{\underset{}{=}}
    P_x(\sup_{t} |F_x(t) - F_n(t)| \leqslant u_{\gamma})
    \xlongrightarrow[]{\text{th. Колмогорова}} \mathcal{K}(u_{\gamma}) = \gamma
  .\end{align*}
\end{proof}

%:: NOTE a:dolzhanskii
%:: NOTE Отсутствует теорема Мизеса-Смирнова
