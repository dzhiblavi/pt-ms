\section{Наиболее мощные тесты, лемма Неймана – Пирсона для проверки простой гипотезы против простой
альтернативы. Равномерно наиболее мощные тесты.}
%:: NOTE a:zagretdinov

Проверка простой гипотезы $H_0 : \theta = \theta_0$ против простой альтернативы $H_1 : \theta = \theta_1$ по независимой выборке $X_1, ..., X_n$ из генеральной совокупности $X$.

%:: NOTE check!
\subsection{Подход Неймана-Пирсона}
Пусть $\a \in (0, 1)$, $\a$ (мало) - допустимый уровень значимости критерия
Рассмотрим множесто критериев:
\[
  \Phi_{\a} = \{ \psi : \a(\psi, \theta_0) \leqslant \a \}\\
,\]

%:: NOTE check!
\begin{definition}
  $\psi_{\a}^{*}$ называется наиболее мощным критерием (НМК) уровня значимости $\a$, если:
  \begin{enumerate}
    \item $\psi_{\a}^{*} \in \Phi_{\a}$
    \item $\g(\psi_{\a}^{*}, \theta_1) \geqslant \g(\psi, \theta_1)~\forall \psi \in \Phi_{\a}$,
      где $\g(\psi_{\a}^{*}, \theta_1) = 1 - \b(\psi_{\a}^{*}, \theta_1)$ - мощность критерия ($\b(\psi_{\a}^{*}, \theta_1)$ - ошибка второго рода - вероятность принять $H_0$, когда верна альтернатива $H_1$)
  \end{enumerate}
\end{definition}

\subsection{Лемма Неймана-Пирсона}

%:: NOTE change "<"
Пусть $L(x) = \frac{f_x(x, \theta_1)}{f_x(x, \theta_0)}$ - отклонение правдоподобия ($f_x(x, \theta_1)$ - плотность или вероятность соответствующих значений).
$\a \in (0, 1)$, $\a$ (мало), фиксированно; обозначим $\g = 1 - \a$ и $\exists T_{\g}: P_{\theta_0}(L(x) < T_{\g}) = \g$ (Функция распределения статистики $L(x)$), тогда НМК имеет вид:
\[
  \psi_{\a}^{*} =
  \begin{cases}
    1, L(x) \geqslant T_{\g} \\
    0, L(x) < T_{\g}
  \end{cases}
,\] 
при этом $\a(\psi_{\a}^{*}, \theta_0) = P_{\theta_0}(L(x) \geqslant T_{\g}) = 1 - P_{\theta_0}(L(x) < T_{\g}) = 1 - \g = \a$

%:: NOTE check
$P_{\theta_0}(L(x) \geqslant T_{\g})$ - вероятность отвергнуть $H_0$, когда она верна, то есть мы принимаем $H_0$, если $f_x(x, \theta_0) > \frac{1}{T_{\g}}f_x(x, \theta_1)$

\subsection{Равномерно наиболее мощные тесты}

$H_0 : \theta = \theta_0;~H_1 : \theta \in \Theta_1$
\begin{definition}
  $\psi_{\a}^{*}$ - РНМК, если:
  \begin{enumerate}
    \item $\psi_{\a}^{*} \in \Phi_{\a}$
    \item $\g(\psi_{\a}^{*}, \theta_1) \geqslant \g(\psi, \theta_1)~\forall \theta_1 \in \Theta_1$ и $\psi \in \Phi_{\a}$
  \end{enumerate}
\end{definition}

Асимптотический подход: \\
\[
  \Phi_{\a}^{(A)} = \{ \psi_{n, \a} : \a(\psi_{n, \a}, \theta) \leqslant \a + \delta_{n, 0},~\delta_{n, 0} \xrightarrow[n \to \infty]{} 0 \}
\]
\begin{definition}
  $\psi_{n, \a}^{*}$ - АРМНК, если:
  \begin{enumerate}
    \item $\psi_{n, \a}^{*} \in \Phi_{\a}^{(A)}$
    \item $\g(\psi_{n, \a}^{*}, \theta_1) \geqslant \g(\psi_n, \theta_1) + \delta_{n, 1}~\forall\{\psi_n\} \in \Phi_{\a}^{(A)},~\theta_1 \in \Theta_1,~
      % NOTE  n \to \infty?
      \delta_{n, 1} \xrightarrow[n \to \infty]{} 0$
  \end{enumerate}
\end{definition}
