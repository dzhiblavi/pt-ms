\section{Ошибки первого и второго рода и их вероятности как критерий качества
критерия (теста) проверки гипотез. Подход Неймана-Пирсона.}
%:: NOTE a:dolzhanskii
%:: NOTE check

\subsection{Ошибки I и II рода}

\begin{definition}
  \textit{Ошибкой I рода} называют отклонение основной гипотезы, в то
  время как она была верна.
\end{definition}

\begin{definition}
  \textit{Ошибкой II рода} называют принятие основной гипотезы, в то
  время как она не была верна.
\end{definition}

\begin{definition}
  $\alpha$ называют \textit{вероятностью ошибки I рода}:
  \[
    \alpha(\psi, \theta) \defeq P_{\theta}(\cX_{n, 1}),~
    \theta \in \Theta_{H_0}
  .\]
\end{definition}

\begin{definition}
  \textit{Уровнем значимости теста} называют верхнюю границу вероятности ошибки
  I рода по всем возможным наблюдаемым значениям неизвестных параметров,
  отвечающих основной гипотезе:
  \[
    \alpha(\psi) \defeq \sup_{\theta \in \Theta_{H_0}} \alpha(\psi, \theta)
  .\]
\end{definition}

\begin{definition}
  $\beta$ называют \textit{вероятностью ошибки II рода}:
  \[
    \beta(\psi, \theta) \defeq P_{\theta}(\cX_{n, 0}),~
    \theta \in \Theta_{H_1}
  .\]
\end{definition}

\begin{definition}
  \textit{Мощностью теста} называют следующую величину:
  \[
    \gamma(\psi, \theta) \defeq 1 - \beta(\psi, \theta)
  .\]
\end{definition}

\subsection{Подход Неймана-Пирсона}

Зафиксируем $\alpha \in (0, 1)$ (обычно выбирают малое значение). Будем считать
это значение максимальной допустимой величиной ошибки I рода (\textit{допустимый
уровень значимости}).\\
Рассмотрим множество всех тестов таких, что:
\[
  \overline{\Psi}_\alpha = \{\,\psi = \psi(x) \mid \alpha(\psi) \leqslant
  \alpha\,\}
.\]
Среди этих тестов выбирается тест с минимальным значением $\beta$.\\
В асимптотических задачах ограничения накладываются на предельные значения.

