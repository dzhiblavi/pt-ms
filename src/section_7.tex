\section{Определение несмещенности и асимптотической нормальности оценки
    характеристики. Построение доверительного интервала для характеристики 
    на базе асимптотической нормальности ее оценки.}
    %:: NOTE a:dzhiblavi
    %:: NOTE check

\begin{definition}
	Напомним, оценка называется \textit{несмещенной}, если
	\[
		b(\hat{g}, \theta) = g(\theta) - E_\theta(\hat{g}) = 0
	.\]
\end{definition}

\begin{definition}
	Последовательность оценок $\hat{g}_n$ называется \textit{асимптотически
	нормальной}, если
	\[
		\sqrt{n} \cdot (\hat{g}_n - g(\theta)) = 
	Y_n \xrightarrow[]{P_{n, \theta}} Y \sim N(0, \Delta^2(\theta)) 
	.\]
\end{definition}

\begin{definition}
	Величина $\Delta(\theta)$ из определения асимптотически нормальной
	оценки называется \textit{нормирующим множителем}. 
\end{definition}

\begin{remark}
	Определение асимптотически нормальной оценки можно переписать так:
	\[
		\frac{\sqrt{n} \cdot (\hat{g}_n - g(\theta))}{\Delta(\theta)}
	\xrightarrow[]{P_{n, \theta}} Y \sim N(0, 1)
	.\]
\end{remark}

На базе асимптотической нормальности можно построить доверительный интервал.
Выпишем определение асимптотической нормальности:
\[
	Y_n = \frac{\sqrt{n} \cdot (\hat{g} - g(\theta))}{\Delta(\theta)} 
    \to N(0, 1)
.\]
Это буквально означает:
\[
	P_{n, \theta}(Y_n < t) \to F_{N(0, 1)}(t) 
.\]
Раскроем определение $Y_n$, возьмем его по модулю и воспользуемся квантилью:
\[
	P_{n, \theta}\parens*{\abs*{\frac{\sqrt{n} 
    \cdot (\hat{g} - g(\theta))}{\Delta(\theta)} } < t_\gamma} \to \gamma \Llra
    P_{n, \theta}\parens*{\frac{\Delta(\theta)}{\sqrt{n}}t_\gamma + \hat{g}
    > g(\theta) > -\frac{\Delta(\theta)}{\sqrt{n}}t_\gamma + \hat{g}} \to \gamma
.\]
Здесь $\gamma = P(|\xi| < t_\gamma),~ \xi \sim N(0, 1)$.

