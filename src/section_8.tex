\section{Постановка задачи проверки гипотез}
%:: NOTE a:dolzhanskii
%:: NOTE check

\begin{definition}
	\textit{Гипотезой} называется множесто предполагаемых зафиксированных значений
	некоторого подмножества неизвестных параметров:
	\[
		H \subseteq \Theta
	.\]
\end{definition}

\begin{definition}
	Гипотезу называют \textit{простой}, если $|H| = 1$.
\end{definition}

\begin{definition}
	Гипотезу называют \textit{сложной}, если $|H| > 1$.
\end{definition}

\begin{definition}
	Гипотезами \textit{согласия} называют набор из двух гипотез: основной $H_0$ и
	альтернативы $H_1$, причем $H_0 = \overline{H_1}$.
\end{definition}

\begin{definition}
	Правило принятия или отклонения основной гипотезы $H_0$ называют
	\textit{тестом (критерием)} проверки гипотезы:
	\[
		\psi \colon \cX_{n} \to \{0, 1\}
	.\]
	При этом:
	\begin{itemize}
		\item $\cX_{n, 0}$ называют \textit{допустимым множеством}.
		\item $\cX_{n, 1}$ называют \textit{критическим множеством}.
		\item $\cX_{n, 0} \sqcup \cX_{n, 1} = \cX_{n}$.
	\end{itemize}
\end{definition}

\begin{definition}
	Случайная величина $L\colon \cX_n \to \R$ называется
	\textit{тестовой статистикой}, если она служит порогом для правила принятия
	или отклонения основной гипотезы:
	\[
        \psi(X^{(n)}) = \begin{cases}
            0,~ L(X^{(n)}) < T \> (H_0) \\
            1,~ L(X^{(n)}) \geqslant T \> (H_0)
		\end{cases}
	.\]
	Где $T$ называют \textit{порогом принятия решения}.
\end{definition}

