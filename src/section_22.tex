\section{Доверительное оценивание и проверка гипотез на основе оценок
максимального правдоподобия.}
%:: NOTE a:dzhiblavi
%:: NOTE check

\subsection{Доверительное оценивание}

Напомним утверждение, сформулированное ранее.

\begin{theorem}
    
    \textit{При достаточно общих условиях} оценка максимального правдоподобия
    одномерного параметра обладает свойством асимптотической нормальности с 
    $\Delta^2(\theta) = \frac{1}{I(\theta)}$.
\end{theorem}

Обладая этой информацией, нетрудно построить доверительный интервал
на основе оценки $\hat{g}$, полученной методом максимального правдоподобия. 
Действительно, мы только что сформулировали тот факт, что
\[
    \sqrt{n} \cdot (\hat{g} - g(\theta)) = Y_n 
    \xrightarrow[n \to +\infty]{P_{n, \theta}} Y \sim N(0, \Delta^2(\theta))
.\]
Откуда вытекает
\[
    P_{n, \theta}(|Y_n / \Delta(\theta)| \leqslant t_\gamma) 
    \to \gamma 
.\]
Здесь $t_\gamma \colon~ P_{N(0, 1)}(|\xi| < t_\gamma) = \gamma,~ \xi 
\sim N(0, 1)$. Чтобы в явном виде получить доверительный интервал, раскроем 
определение $Y_n$ и подставим значние $\Delta(\theta)$:
\begin{align*}
          &P_{n, \theta}\parens*{\abs*{\sqrt{n} \cdot (\hat{g} - g(\theta)) \cdot 
          \sqrt{I(\theta)}} \leqslant t_\gamma} \to \gamma \\
    \Llra &P_{n, \theta}(\hat{g} - \delta \leqslant g(\theta) \leqslant
    \hat{g} + \delta) \to \gamma,~ \delta = \frac{t_\gamma}{\sqrt{n I(\theta)}}
.\end{align*}

\subsection{Проверка гипотез}
Пусть сформулирована гипотеза $H_0 \equiv \theta = \theta_0$ и 
альтернатива $H_1 \equiv \theta = \theta_1 > \theta_0$ (Такая
альтернатива называется \textit{правосторонней}). Пусть также
имеется оценка $\hat{\theta}$, полученная методом максимального правдоподобия.
Рассмотрим тест:
\[
    \psi^*_{n, \a} = \begin{cases}
        1,~ \sqrt{n I(\theta)} \cdot (\hat{\theta} - \theta_0) \geqslant 
        c_{1 - \a} \\
        0,~ \sqrt{n I(\theta)} \cdot (\hat{\theta} - \theta_0) < c_{1 - \a}
    \end{cases}
.\]
Здесь $c_\gamma \colon~ P(\xi < c_\gamma) = \gamma,~ \xi \sim N(0, 1)$.

\begin{theorem}
    $\psi^*_{n, \a}$ является состоятельным тестом асимптотического уровня 
    значимости $\a$.
\end{theorem}
\begin{proof}
    \enewline
    \begin{itemize}
        \item Вычислим уровень значимости критерия:
            \[
                \a(\psi^*_{n, \a}) = P_{n, \theta_0}\parens*{\sqrt{n I(\theta)} 
                \cdot (\hat{\theta} - \theta_0) \geqslant c_{1 - \a}} = 
                1 - P_{n, \theta_0}\parens*{\sqrt{n I(\theta)} 
                \cdot (\hat{\theta} - \theta_0) < c_{1 - \a}} \to \a
            .\]
            Последнее верно по свойству асимптотической нормальности оценки.
        \item Проверим состоятельность теста:
%:: NOTE самопальное доказательство, проверьте плиз!
            \begin{align*}
                \b(\psi^*_{n, \a}) 
                &= P_{n, \theta_1}\parens*{\sqrt{n I(\theta)} 
                \cdot (\hat{\theta} - \theta_0) < c_{1 - \a}} \\ 
                &= P_{n, \theta_1}\parens*{\sqrt{n I(\theta)} \cdot
                    (\hat{\theta} - \theta_1) < \underbrace{\sqrt{n I(\theta)} 
                \cdot (\theta_0 - \theta_1)}_{\to -\infty} + c_\gamma}
                \xrightarrow[n \to +\infty]{} 0
            .\end{align*}
    \end{itemize}
\end{proof}

