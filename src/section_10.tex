\section{Асимптотический вариант задачи проверки гипотез. Состоятельный тест
асимптотического уровня значимости $\alpha$.}
%:: NOTE a:dolzhanskii
%:: NOTE check

При асимптотическом подходе последовательность тестов $\psi = \psi_n$
называют просто тестом и проводят исследование асимптотических (предельных)
свойств тестов $\psi = \{\psi_n\}$ при $n \to \infty$.\\

\begin{definition}
  Тест $\psi = \{\psi_n\}$ имеет \textit{асимптотический уровень
  значимости $\alpha(\psi),~ \alpha(\psi) \in \left[0, 1 \right]$}, если:
  \[
    \alpha_n(\psi_n) =
    \sup_{\theta \in \Theta_{H_0}} \alpha(\psi_n, \theta) \to
    \alpha(\psi),~ n \to \infty
  .\]
\end{definition}

При использовании подхода Неймана-Пирсона в асимптотическом варианте ограничение
накладывается на асимптотический уровень значимости: $\alpha(\psi) = \alpha$.

\begin{definition}
  При асимптотическом подходе тест $\psi = \{\psi_n\}$ называется
  \textit{состоятельным}, если для любого $\theta \in \Theta_{H_1}$:
  \[
    \beta(\psi_n, \theta) \xlongrightarrow[n \to \infty]{} 0
  .\]
\end{definition}

