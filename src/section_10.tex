\section{Асимптотический вариант задачи проверки гипотез. Состоятельный тест
асимптотического уровня значимости $\alpha$.}
%:: NOTE a:dolzhanskii
%:: NOTE check

При асимптотическом подходе последовательность тестов $\varphi = \varphi_n$
называют просто тестом и проводят исследование асимптотических (предельных)
свойств тестов $\varphi = \{\varphi_n\}$ при $n \to \infty$.\\

\begin{definition}
  Тест $\varphi = \{\varphi_n\}$ имеет \textit{асимптотический уровень
  значимости $\alpha(\varphi),~ \alpha(\varphi) \in \left[0, 1 \right]$}, если:
  \[
    \alpha_n(\varphi_n) =
    \sup_{\theta \in \Theta_{H_0}} \alpha(\varphi_n, \theta) \to
    \alpha(\varphi),~ n \to \infty
  .\]
\end{definition}

При использовании подхода Неймана-Пирсона в асимптотическом варианте ограничение
накладывается на асимптотический уровень значимости: $\alpha(\varphi) = \alpha$.

\begin{definition}
  При асимптотическом подходе тест $\varphi = \{\varphi_n\}$ называется
  \textit{состоятельным}, если для любого $\theta \in \Theta_{H_1}$:
  \[
    \beta(\varphi_n, \theta) \xlongrightarrow[n \to \infty]{} 0
  .\]
\end{definition}

%:: NOTE a:dolzhanskii
%:: NOTE Возможно, это нужно не здесь

\begin{definition}
  \textit{Мерой близости альтернативы $\theta \in \Theta_{H_1}$ и гипотезы
  $H_0$} называют следующую величину:
  \[
    \rho(\theta, \Theta_{H_0}) = \inf_{\theta_0 \in \Theta_{H_0}}
    \left\lVert \theta - \theta_0 \right\rVert
  .\]
\end{definition}

\begin{definition}
  Тест $\varphi = \{\varphi_n\}$ называется \textit{$\sqrt{n}$-состоятельным},
  если:
  \[
    \beta(\varphi_n, \theta_n) \xlongrightarrow[n \to \infty]{} 0
  \]
  %:: NOTE a:zagretdinov
  %:: NOTE "такой" а не для "любой"?
  Для такой последовательности $\theta_n \in \Theta_{H_1}$, что:
  \[
    \sqrt{n} \rho(\theta_n, \Theta_{H_0}) \to \infty
  .\]
\end{definition}
