\section{Постановка задач математической статистики}
%:: NOTE a:dzhiblavi
%:: NOTE check

\textit{Сравним задачи теории вероятностей и математической статистики}

\subsection{Задачи теории вероятностей}
Заданы:
\begin{itemize}
	\item Вероятностное пространство $\langle \O, \Sigma, P \rangle$.
	\item Случайная величина $X \colon \O \to \R^n$.
\end{itemize}
Требуется получить различного рода характеристики величины $X$ и величин
, получающихся из $X$.

\subsection{Задачи математической статистики}
\begin{definition}
	\textit{Статистическим экспериментом} называется четверка
	\[
		\langle \cX, \cA, P_{\theta}, \Theta \rangle
	.\]
	Здесь:
	\begin{itemize}
		\item $\cX$ -- множество \textit{наблюдений}.
		\item $\cA$ -- $\sigma$-алгебра подмножеств $\cX$.
			%:: NOTE и зачем оно?
		\item $P_{\theta}$ -- известная с точностью до неизвестного параметра
			$\theta$ вероятностная мера -- закон распределения наблюдаемых
			данных.
		\item $\Theta$ -- множество допустимых значений неизвестного параметра,
			то есть $\theta \in \Theta$.
	\end{itemize}
\end{definition}

\textit{Задачей математической статистики является получение той или иной
информации о законе распределения наблюдаемых данных $P = P_\theta$.}

\begin{definition}
	\textit{Статистикой} называется измеримая функция
	\[
		f \colon \cX \to A
	.\]
	Для произвольного $A$.
\end{definition}

\begin{definition}
	Пусть
	\[
		\overline{X} = \langle X_1, \ldots, X_n \rangle
	.\]
	Где $X_i \sim X$ -- одинаково распределенные случайные величины.
	Соответствующая модель называется \textit{моделью независимой
	однородной выборки}.
\end{definition}

\begin{definition}
	\textit{Гипотезой} $H$ называется подмножество $\Theta$:
	\[
		H \subseteq \Theta
	.\]
\end{definition}

Перечислим некоторые задачи математической статистики.
\begin{itemize}
	\item Оценивание параметра $\theta$ или какой-либо функции $g(\theta)$,
		то есть построение статистики $\hat{g} \colon \cX \to \Theta$.
		Оценивание может быть:
		\begin{itemize}
			\item \textit{точечным}, то есть указание численной оценки
				$g(\theta)$
			\item \textit{доверительным}, то есть указание множества, с
				фиксированной вероятностью содержащего $g(\theta)$
		\end{itemize}
	\item Проверка гипотез. Пусть имеется разбиение $\Theta$ на гипотезы:
		$\Theta = \bigsqcup_{n \in N}{H_n}$. Тогда проеркой гипотезы назовем
		построение \textit{теста} (\textit{критерия}), то есть отображения
		\[
			\psi \colon \cX \to N
		.\]
		Которое по наблюдению выдает номер гипотезы, которому это наблюдение
		``соответствует''.
\end{itemize}

Естественно, перечисленные задачи можно оценивать с точки зрения качества.
В этом смысле всегда требуется с точки зрения какой-либо метрики  построить
``лучшую'' оценку.

