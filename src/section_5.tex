\section{Вид квадратичного риска в случае одномерной характеристики.}
%:: NOTE a:dzhiblavi
%:: NOTE check

\begin{definition}
    \textit{Смещением} оценки называется величина
    \[
        b(\hat{g}, \theta) = g(\theta) - E_\theta(\hat{g})
    .\]
\end{definition}

\begin{definition}
    Оценка называется \textit{несмещенной}, если $b(\hat{g}, \theta) = 0$.
\end{definition}

\begin{theorem}
    $R_2(\hat{g}, \theta) = D_\theta(\hat{g}) + b^2(\hat{g}, \theta)$.
\end{theorem}
\begin{proof}
    \begin{align*}
        R_2(\hat{g}, \theta) 
        &= E_\theta(\norm{\hat{g} - g(\theta)}^2) =
        E_\theta(\hat{g} - E_\theta(\hat{g}) - (g(\theta) -
        E_\theta(\hat{g})))^2 \\
        &= E_\theta(\hat{g} - E_\theta(\hat{g}))^2 + (g(\theta) - 
        E_\theta(\hat{g}))^2 - \underbrace{2(g(\theta) - E_\theta(\hat{g}))
        (E_\theta{\hat{g}} - E_\theta{\hat{g}})}_0 \\
        &= D_\theta(\hat{g}) + b^2(\hat{g}, \theta)
    .\end{align*}
    \end{proof}

    \begin{corollary}
        Для одномерных несмещенных оценок квадратичный риск в точности равен
        дисперсии оценки:
        \[
            R_2(\hat{g}, \theta) = D_\theta(\hat{g})
        .\]
    \end{corollary}

