\section{О сравнении качества оценок. Свойства функции правдоподобия (одномерный параметр). Неравенство Рао-Крамера и эффективные оценки.}
%:: NOTE a:koveshnikov
%:: NOTE check

%:: NOTE \chi -> \cX
Рассматриваются задачи оценки конечномерного параметра распределения $P_{\theta}$, $\theta \in \Theta \subset R^m$, а также характеристик (функций) $g(\theta)$ по наблюдениям $X \in \cX$.

\subsection{О сравнении качества оценок}
Сравниваем различные оценки с помощью функции риска. 
\begin{definition}
	Оценка $\hat{g}^{1}$ не хуже оценки $\hat{g}^{2}$, если $R(\hat{g}^{1}, \theta) \leq R(\hat{g}^{2}, \theta)$, для всех $\theta \in \Theta$. Обозначение: $\hat{g}^{1} \succeq \hat{g}^{2}$.
\end{definition}

\begin{definition}
	Пусть $G = \{\,\hat{g}\,\}$ -- некоторый класс оценок. Оценка $\hat{g}^{*}$ называется \textit{эффективной} в классе $G$, если:
	\[
%:: NOTE так нельзя!!!
        \hat{g}^{*} \succeq \hat{g}, \text{ для всех } \hat{g} \in G
	.\]
\end{definition}

В классе \textit{всех} оценок не существует эффективной оценки. Для поиска эффективных оценок нужны ограничения на класс рассматрвиаемых оценок. 

Отмеченные трудности вынуждают сравнивать не функции риска различных оценок, а какие-нибудь числовые величины от функций риска, которые характеризуют функцию риска. 

\subsubsection{Минимаксный подход}
Здесь качество оценки характеризуется максимальным значением риска:
%:: NOTE max -- \max
\[
    R_{\max}(\hat{g}) = \sup_{\theta \in \Theta} R(\hat{g}, \theta)
.\]

\begin{definition}
	Оценка $\hat{g}$ называется \textit{минимаксной}, если:
	\[
        R_{\max}(\hat{g}) \leqslant R_{\max}(\tilde{g}) \text{ для любой оценки } \tilde{g}
	.\]
\end{definition}

Подход ориентирован на построение оценки с минимальным значением максимального риска.

\subsubsection{Асимптотически минимаксные оценки}
\begin{definition}
	Оценка $\hat{g}_{n}$ называется \textit{асимптотически минимаксной}, если:
	\[
%:: NOTE () -- \parens*{}
		\lim_{n \to +\infty}\parens*{\frac{R_{\max}(\hat{g}_n)}{R_{\max}(\tilde{g}_n)}} \leqslant 1
	.\]
\end{definition}

%:: NOTE todo
\subsubsection{Асимптотически минимаксные оценки}
\begin{definition}
	Оценка $\hat{g}_{n}$ называется \textit{локально асимптотически минимаксной} в точке $\theta_{0} \in \Theta$, если:
	\[
		\lim_{n \to +\infty}\parens*{\frac{todo}{todo}} \leqslant 1
	.\]
\end{definition}

%:: NOTE todo
\subsection{Свойства функции правдоподобия (одномерный параметр)}
TODO

\subsection{Неравенство Рао-Крамера и эффективные оценки}
Пусть $\hat{\theta} = \hat{\theta}(X)$ -- несмещенная оценка одномерного параметра, выполнены условия регулярности и $I(\theta) > 0$ для всех $\theta \in \Theta$.

\begin{theorem}(Неравенство Рао-Крамера)

	Для любого $\theta \in \Theta \subset R^{1}$
	\[
%:: NOTE fraq?!
		D_{\theta}(\hat{\theta}) \geqslant \frac{1}{I(\theta)}
	.\]
\end{theorem}
%:: NOTE todo
\begin{proof}
	\enewline
        \begin{itemize}
		\item TODO
	\end{itemize}
\end{proof}

Неравенство Рао-Крамера дает нижнюю границу для дисперсии и квадратичного риска несмещенных оценок.

\begin{definition}
	Оценка, на которой достигается нижняя граница Рао-Крамера называется \textit{эффективной}.
\end{definition}

\begin{remark}
	Неравенство Рао-Крамера справедливо и для смещенных оценок (со смещением $b^{'}_{\theta}(\hat{\theta})$) в форме:
	\[
		D_{\theta}(\hat{\theta}) \geqslant \frac{(1 + b^{'}_{theta}(\hat{\theta}))^2}{I(\theta)}
	.\]
\end{remark}

