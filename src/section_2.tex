\section{Частота как оценка вероятности события и её свойства.
    Построение доверительного интервала для вероятности события
на базе асимптотической нормальности частоты.}
%:: NOTE a:dzhiblavi
%:: NOTE check

\begin{theorem}(Яков, Бернулли)

    Пусть имеется $\xi_i \sim \xi$ -- последовательность одинаково 
    распределенных и попарно независимых случайных величин. Пусть
    \[
        \overline{\xi}_n = \frac{1}{n} \sum_{i = 1}^n{\xi_n} = \frac{k_n}{n}
    .\]
    Тогда
    \[
        \overline{\xi}_n \underset{n \to +\infty}{\rcon} p
    .\]
\end{theorem}

\begin{theorem}(Центральная предельная теорема, простейший вариант)

    Пусть случайные величины $X_i \sim X$ независимы и одинаково распределены,
    причем $\exists E(X), D(X)$. Тогда для случайной величины
    \[
        Y_n = \frac{\overline{X}_n - E\parens*{\overline{X}_n}}
        {\sigma\parens*{\overline{X}_n}}
    .\]
    Верно:
    \[
        F_{Y_n} \underset{\R}{\rcon} F_{N(0, 1)}
    .\]
\end{theorem}

\begin{theorem}(Свойства частоты как оценки $p$)

    Пусть $\xi \sim B(p)$. Тогда 
    \[
        \hat{p} = \frac{k_n}{n}
    \]
    Является несмещенной асимптотически нормальной оценкой $p$, то есть
    \[
        E(\hat{p}) = p,
    \]
    \[
        \sqrt{n} \cdot (\hat{p} - p) = Y_n \xrightarrow[]{P_{n, \theta}} Y 
        \sim N(0, \Delta^2(p)),~ \Delta^2(p) = p(1 - p)
    .\]
\end{theorem}
\begin{proof}
    \enewline
    \begin{itemize}
        \item Покажем несмещенность:
            \[
                E(\hat{p}) = E\parens*{\frac{k_n}{n}} = \frac{1}{n} np = p
            .\]
        \item Асимптотическая нормальность с нормирующим множителем
            $\Delta^2(p) = p(1 - p)$ следует непосредственно из
            центральной предельной теоремы.
    \end{itemize}
\end{proof}

На базе асимптотической нормальности можно построить доверительный интервал.
Проделаем это на примере частоты. Выпишем определение асимптотической
нормальности:
\[
    Y_n = \frac{\sqrt{n} \cdot (\hat{p} - p)}{\sqrt{p(1 - p)}} \to N(0, 1)
.\]
Это буквально означает:
\[
    P_{n, \theta}(Y_n < t) \to F_{N(0, 1)}(t) 
.\]
Раскроем определение $Y_n$, возьмем его по модулю и воспользуемся квантилью:
\[
    P_{n, \theta}\parens*{\abs*{\frac{\sqrt{n} 
    \cdot (\hat{p} - p)}{\sqrt{p(1 - p)}} } < t_\gamma} \to \gamma \Llra
    P_{n, \theta}\parens*{\frac{\sqrt{p(1 - p)}}{\sqrt{n}}t_\gamma + \hat{p}
    > p > -\frac{\sqrt{p(1 - p)}}{\sqrt{n}}t_\gamma + \hat{p}} \to \gamma
.\]
Здесь $\gamma = P(|\xi| < t_\gamma),~ \xi \sim N(0, 1)$.
Построим д

