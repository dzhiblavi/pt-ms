\section{Функции потерь и функции риска, состоятельность оценки характеристики, 
достаточное условие для состоятельности оценки.}
%:: NOTE a:dzhiblavi
%:: NOTE check

\begin{definition}
	\textit{Оценкой} $g(\theta)$ называется статистика вида
	\[
		\hat{g} \colon \cX \to g(\Theta)
	.\]
\end{definition}

\begin{definition}
	Пусть $\hat{g}(\theta)$ -- оценка $g(\theta)$. Тогда \textit{функцией
	потерь} называется неотрицательная функция $l(\hat{g}, g(\theta))$, 
	характеризующая ``близость'' оценки к настоящему значению.
\end{definition}

\begin{remark}
	Обычно в качестве функции потерь рассматривают функцию вида
	\[
		l(\hat{g}, g(\theta)) = \omega(\norm{\hat{g}, g(\theta)})
	.\]
	Здесь $\omega$ -- неотрицательная монотонно возрастающая функция,
	$\omega(0) = 0$.
\end{remark}

\begin{remark}
	$l$ являтся случайной величиной.
\end{remark}

\begin{definition}
	\textit{Риском} называется функция
	\[
		R(\hat{g}, \theta) \defeq E_{\theta}(l(\hat{g}, g(\theta))) 
	.\]
\end{definition}

\begin{remark}
	Риск -- функция параметра $\theta$ и способа оценивания $\hat{g}$.
\end{remark}

Опишем самые важные для нас виды функции потерь и риска.

\begin{definition}
	Определим функцию потерь индикатором отклонений:
	\[
		l^\delta(\hat{g}, g(\theta)) = \omega^\delta(\norm{\hat{g}, g(\theta)})
	.\] 
	Где
	\[
		\omega(t) = \mathbb{1}_\delta(t) = \begin{cases}
			0,~ t < \delta  \\
			1,~ t \geqslant \delta
		\end{cases}
	.\]
	Соответствующий риск будет вероятностью отклонения:
	\[
		R^\delta(\hat{g}, \theta) = E_{\theta}(l(\hat{g}, g(\theta))) =
	0 \cdot P_\theta(\norm{\hat{g} - g(\theta)} < \delta) +
	1 \cdot P_\theta(\norm{\hat{g} - g(\theta)} \geqslant \delta) =
	P_\theta(\norm{\hat{g} - g(\theta)} \geqslant \delta)
	.\]
\end{definition}

\begin{definition}
	При асимптотическом подходе оценка называется \textit{состоятельной},
	если 
	\[
		\forall \delta > 0~ R^\delta(\hat{g}_n, \theta) =
	P_{n, \theta}(\norm{\hat{g}_n - g(\theta)} \geqslant \delta)
	\xrightarrow[n \to +\infty]{} 0
	.\]
	Или, что то же самое:
	\[
		\hat{g}_n \xrightarrow[n \to +\infty]{P_{n, \theta}} g(\theta)
	.\]
\end{definition}

\begin{definition}
	\textit{Квадратичной функцией потерь} называется функция
	\[
		l_2(\hat{g}, g(\theta)) = \norm{\hat{g} - g(\theta)}^2
	.\]
	Соответствующий ей риск называется \textit{квадратичным}:
	\[
		R_2(\hat{g}, \theta) = E_\theta(\norm{\hat{g} - g(\theta)}^2)
	.\]
\end{definition}

\begin{theorem}(Достаточное условие для состоятельности оценки)

	$R_2(\hat{g}_n, \theta) \xrightarrow[n \to +\infty]{} 0 \Lra$ оценка
	состоятельна.
\end{theorem}
\begin{proof}
	\begin{align*}
		\forall \delta > 0~ R^\delta(\hat{g}_n, \theta)
		&= P(\norm{\hat{g}_n - g(\theta)} \geqslant \delta) =
		P(\norm{\hat{g}_n - g(\theta)}^2 \geqslant \delta^2)  \\
		&\leqslant \frac{E_\theta(\norm{\hat{g}_n - g(\theta)}^2)}{\delta^2} =
		\frac{R_2(\hat{g}_n, \theta)}{\delta^2} \xrightarrow[n \to +\infty]{} 0
	.\end{align*} 
\end{proof}

