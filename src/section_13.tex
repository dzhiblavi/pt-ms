\section{Критерии согласия Колмогорова и Мизеса-Смирнова.}

Пусть $F_0(t)$ -- заданная непрерывная функция распределения.\\

Поставим задачу проверки согласия:
\[
  H_0 \equiv (F_x(t) \equiv F_0(t))
.\]

\subsection{Критерий согласия Колмогорова}

Определим тестовую статистику:
\[
  L(\overline{X}) = \sqrt{n} \rho_{\infty}(F_0, F_n)
.\]

По th. Колмогорова:
\[
  P(L(\overline{X}) < z) \xrightarrow[n \to \infty]{} \mathcal{K}(z)
,\]
где $\mathcal{K}$ -- распределение Колмогорова.
Тогда порогом принятия решения при уровне значимости $\alpha$ является
квантиль распределения Колмогорова порядка $1 - \alpha$
(далее $u_{1 - \alpha}$).\\

Таким образом, определим тест:
\[
  \varphi(\overline{X}) = \begin{cases}
    0,~ \sqrt{n} \rho_{\infty}(F_0, F_n) < u_{1 - \alpha}\\
    1,~ \sqrt{n} \rho_{\infty}(F_0, F_n) \geqslant u_{1 - \alpha}
  \end{cases}
.\]

\subsection{Критерий Мизеса-Смирнова}

Определим тестовую статистику:
\[
  L(\overline{X}) = \sqrt{n} \rho_{2}(F_0, F_n)
.\]

По th. Мизеса-Смирнова:
\[
  P(L(\overline{X}) < z) \xrightarrow[n \to \infty]{} \mathcal{S}(z)
,\]
где $\mathcal{S}$ -- распределение Мизеса-Смирнова.
Тогда порогом принятия решения при уровне значимости $\alpha$ является
квантиль распределения Мизеса-Смирнова порядка $1 - \alpha$
(далее $s_{1 - \alpha}$).\\

Таким образом, определим тест:
\[
  \varphi(\overline{X}) = \begin{cases}
    0,~ \sqrt{n} \rho_{2}^{2}(F_0, F_n) < w_{1 - \alpha}\\
    1,~ \sqrt{n} \rho_{2}^{2}(F_0, F_n) \geqslant w_{1 - \alpha}
  \end{cases}
.\]

\pagebreak

\subsection{Прикладной алгоритм}
\begin{enumerate}
  \item Строится ЭФР.
  \item Считается статистика критерия. Поскольку ЭФР является
  кусочно-постоянной, расстояние Колмогорова / Мизеса-Смирнова можно считать как
  верхнюю границу по соответствующим значениям расстояний в точках скачка.
  \item Для заданного уровня значимости $\alpha$ находится квантиль
  распределения Колмогорова / Мизеса-Смирнова порядка $1 - \alpha$.
  \item Если значение тестовой статистики меньше полученного квантиля,
  следует принять нулевую гипотезу, иначе -- оклонить.
\end{enumerate}
