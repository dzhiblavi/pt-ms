\section{Постановка выборочной статистической модели. Точечная оценка параметра
  и характеристики.}
  %:: NOTE a:dolzhanskii
  %:: NOTE check

\begin{definition}
	Напомним, что \textit{точечной оценкой} параметра $\theta$ или какой-либо
	функции $g(\theta)$ называют численную оценку этой величины.
\end{definition}

\textit{Пусть $\hat{g}$ является некоторой точечной оценкой $g = g(\theta)$.}

\begin{definition}
	$\hat{g}$ называется \textit{несмещенной}, если $E(\hat{g}) = g(\theta).$
\end{definition}

\begin{definition}
	При асимптотическом подходе оценка $\hat{g}$ называется 
    \textit{состоятельной}, если $\hat{g} \xLongrightarrow[P]{} g(\theta)$ при 
    $n \to \infty$.
\end{definition}

\begin{definition}
	$\hat{g}_n$ называется \textit{асимптотически нормальной}, если
	\[
		\frac{\sqrt{n}(\hat{g}_n - g(\theta))}{\sigma(g(\theta))}
        \xrightarrow[]{P_{n, \theta}} N(0, 1)
	.\]
\end{definition}

%:: NOTE в билете 19 более подробное определение
\begin{definition}
	$\hat{g}_n$ называется \textit{эффективной} в классе оценок $K$, если для
	любой другой оценки $\hat{g}^*_n \in K$ имеет место неравенство:
	\[
		E(\hat{g}_n - g(\theta))^2 \leqslant E(\hat{g}^*_n - g(\theta))^2
	.\]
\end{definition}
