\section{Метод максимального правдоподобия и его свойства.}

Будем основывать метод на \textit{принципе максимального правдоподобия}:
в качестве оценки неизвестного параметра распределения выберем то значение,
при котором вероятность наблюдаемых величин наиболее вероятна.

Будем считать, что выполнено одно из двух:
\begin{itemize}
    \item Распределение генеральной совокупности абсолютно непрерывно, то
        есть существует непрерывная плотность, задающая это распределение:
        \[
            f(x, \theta) \Llra P_\theta
        .\]
    \item Распределение дискретно. В таком случае будем обозначать
        \[
            f(x, \theta) = P_\theta(X = x)
        .\]
\end{itemize}

\begin{definition}
    \textit{Функцией правдоподобия} называется функция
    \[
        L(\theta, X) = f(X, \theta)
    .\]
\end{definition}

\begin{definition}
    \textit{Логарифмической функцией правдоподобия} называется функция
    \[
        l(\theta, X) = \ln{L(\theta, X)} = \ln{f(X, \theta)}
    .\]
\end{definition}

\begin{remark}
    При фиксированном $X \in \cX$ функции правдоподобия -- просто вещественные
    функции $\theta$. Если же считать $X$ случайной величиной, то
    и функции правдоподобия становятся случайными величинами. 
\end{remark}

\begin{remark}
    В модели независимой однородной выборки функции правдоподобия принимают вид:
    \[
        L\parens*{\theta, X^{(n)}} = \prod_{i = 1}^n{f\parens*{X^{(n)}, \theta}},~
        l\parens*{\theta, X^{(n)}} = \sum_{i = 1}^n{\ln{f\parens*{X^{(n)}, \theta}}}
    .\]
\end{remark}

\begin{definition}
    \textit{Оценкой максимального правдоподобия} называется значение
    \[
        \theta^*(X) = \argmax_{\theta \in \Theta}{L(\theta, X)}
    .\]
\end{definition}

\begin{definition}
    В случае, когда логарифмическая функция правдоподобия непрерывно 
    дифференцируема, система уравнений
    \[
        \pderv{l(\theta, X)}{\theta_j} = 0
    \]
    Называется \textit{уравнениями максимального правдоподобия}. В этом случае
    $\theta^*(X)$ является одним из решений этой системы.
\end{definition}

\begin{definition}
    \textit{Информацией Фишера} называется функция
    \[
        I(\theta) = E_\theta(l'(\theta, X))^2
    .\]
\end{definition}

\begin{remark}
    Информация Фишера -- числовая характеристика распределения, и не является
    случайной величиной.
\end{remark}

\begin{definition}
    Оценка нызвается $\a(n)$-несмещенной, если
    \[
        \a(n) b_{n, \theta}(\hat{g}_n) \xrightarrow[n \to +\infty]{} 0
    .\]
\end{definition}

\begin{theorem}(Свойства оценки максимального правдоподобия)
    
    Пусть справедливы условия:
    \begin{itemize}
        \item $\theta \in \Theta = \langle a, b \rangle \subseteq \R$, 
            то есть изучаемый параметр одномерный.
        \item $\cX = \R$.
        \item Почти везде существуют частные производные логарифмической
            функции правдоподобия порядка $k \leqslant 3$.
        \item Выполнены неравенства
            \[
                \abs*{\hderv{k}{l(\theta, X)}{\theta^k}} \leqslant G_k(x),~ 1 
                \leqslant k \leqslant 3 
            .\]
            Причем $G_k$ суммируемы и 
            \[
                \sup_{\theta \in \Theta}{\intl_\R{G_3(x) f(x, \theta) \dd x}} 
                < +\infty
            .\]
        \item $\forall \theta > 0~ \exists I(\theta) > 0$.
    \end{itemize}
    Тогда соответстующая оценка максимального правдоподобия обладает свойствами:
    \begin{itemize}
        \item Состоятельность.
        \item $\sqrt{n}$-несмещенность.
        \item Асимптотическая нормальность с $\displaystyle \Delta^2(\theta) = 
            \frac{1}{I(\theta)}$.
    \end{itemize}
\end{theorem}

%:: NOTE примеры.

