\section{Метод моментов и его свойства.}
%:: NOTE a:dzhiblavi
%:: NOTE check

\subsection{Идея метода подстановки}
Метод подстановки уже использовался нами в следующих задачах:
\begin{itemize}
    \item Оценка характеристик распределения $g(F)$ через характеристики
        выборочного распределения $g(F_n)$.
    \item Если $\hat{\theta}_n$ -- в определенном смысле хорошая оценка
        параметра распределения $\theta$, мы используем в качестве оценки
        $g(\theta)$ значение $g(\hat{\theta}_n)$. (\textit{подставляем
        вместо $\theta$ $\hat{\theta}$}).
\end{itemize}

К этому методу можно подойти и с другой стороны.

\subsection{Метод моментов}
Пусть мы ищем параметр распределения $\theta$, причем его можно задать как
решение уравнения
\[
    E_\theta(H(X, \theta)) = 0
.\]
Здесь $H \colon \R \to \R$ -- известная нам функция. Метод состоит в том,
чтобы \textit{заменить} математическое ожидание его выборочной оценкой,
то есть в качестве оценки параметра $\hat{\theta}\parens*{X^{(n)}}$ взять
решение уравнения
\[
    \sum_{i = 1}^n{H(X_i, \theta)} = 0
.\]

\textit{Сформулируем эти идеи в более общем виде.}

Пусть распределение генеральной совокупности $F_X$ известно нам с точностью до
неизвестного параметра $\theta \in \Theta \subseteq \R^m$. Понятно, что все числовые
характеристики распределения $g(F_X)$ можно выразить через неизвестный нам
параметр $\theta$: $g(F_X) = g(\theta)$. Пусть выбранная нами характеристика
$g \colon \R^m \to \R^k$ удовлетворяет следующим свойствам:
\begin{itemize}
    \item Система уравнений относительно $\theta$:
        \[
            g_i(\theta) = g^0_i,~ i = 1..k
        ,\]
        где $g^0 \in \R^k$ -- теоретическое значение характеристики,
        имеет единственное решение.
    \item Система уравнениий обладает свойством \textit{устойчивости}, то есть
        отображение, ставящее в соответствие $g^0$ решение непрерывно в
        окрестности $g^0$.
\end{itemize}
В таком случае, заменим $g^0$ его выборочным аналогом $\hat{g}^0_n$. Решим
ту же самую систему уравнений:
\[
    g_i(\theta) = \hat{g}^0_n,~ i = 1..k
.\]
Остается просто взять в качестве оценки неизвестного параметра $\theta$
найденное нами решение $\hat{\theta}_n$.

\begin{theorem}(Свойства метода моментов)

    Из определения метода моментов сразу вытекают его основные свойства.
    \begin{itemize}
        \item Если $\hat{g}_n$ -- состоятельные оценки, то $\hat{\theta}$
            -- состоятельная оценка.
        \item Аналогичное утверждение справедливо и для свойства
            асимптотической нормальности.
    \end{itemize}
\end{theorem}
\begin{proof}
    \enewline
    \begin{itemize}
        \item Это свойство -- непосредственное следствие устойчивости системы.
        \item Асимптотическая нормальность $\hat{g}_n$ означает
            \[
                \hat{g}_n = g(\theta) + n^{-\frac{1}{2}} Y_n,~
                Y_n \xrightarrow[n \to +\infty]{P_{n, \theta}} Y \sim
                N(0, \mathcal{K}(\theta))
            .\]
            Здесь $\mathcal{K}(\theta)$ -- матрица ковариаций. Чтобы доказать
            утверждение, нам достаточно представить оценку в виде
            \[
                \hat{\theta}_n = \theta + n^{-\frac{1}{2}} Z_n
            .\]
            Где $Z_n \sim N(0, \_)$. По формуле Тейлора:
            \[
                g\parens*{\theta + n^{-\frac{1}{2}} Z_n} =
                g(\theta) + g'(\theta) n^{-\frac{1}{2}} Z_n + \cO\parens*{n^{-1}}
            .\]
            С другой стороны, поскольку $\hat{\theta}_n$ является решением
            соответствующей системы уравнений:
            \[
                g(\hat{\theta}) = \hat{g}_n = g(\theta) + n^{-\frac{1}{2}} Y_n
            .\]
            Приравнивая правые части последних двух уравнений, получаем
            \[
                Z_n \approx (g'(\theta))^{-1} Y_n \xrightarrow[n \to +\infty]
                {P_{n, \theta}} Z \sim N(0, R(\theta))
            .\]
            Где
            \[
                R(\theta) = (g'(\theta))^{-1} \mathcal{K}(\theta)
                (g'(\theta)^\top)^{-1}
            .\]
    \end{itemize}
\end{proof}

%:: NOTE примеры.
