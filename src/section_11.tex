\section{Эмпирическая функция распределения (ЭФР). Построение, свойства ЭФР при
фиксированном значении аргумента (использовать свойства частоты).}
%:: NOTE a:dolzhanskii
%:: NOTE check

\begin{definition}
    \textit{Эмпирической функцией распределения (ЭФР)} называют следующую оценку
    функции распределения генеральной совокупности:
%:: NOTE a:dolzhanskii
%:: NOTE Может есть идеи как это более удачно обозначить?
%:: NOTE dzhiblavi: думаю норм обозначение 
    \[
        F_n(t) = \frac{1}{n} \sum_{i = 1}^{n} \mathbb{1}_{(-\infty, t)}
    .\]
    Иными словами, значение ЭФР в точке $t$ равно отношению числа наблюдений,
    меньших $t$, к их общему числу $n$.
\end{definition}

\paragraph{Свойства ЭФР:}

\begin{enumerate}
    \item ЭФР кусочно-постоянна.
    \item Скачки ЭФР имеют вид $\frac{k}{n}$ для некоторого $k \in (1; n)$.
    \item Область принимаемых значений: $[0; 1]$.
    \item Частота может служить как оценка функции распределения генеральной
        совокупности. При фкисированном $t = t_0$:
        \[
            F_x(t_0) \approx F_n(t_0) = \xi_1 + \ldots + \xi_n =
            \frac{k_n}{n}~ \text{-- частота}
        .\]
    \item $F_n(t)$ является состоятельной оценкой:
        \[
            F_n(t_0) = \overline{\xi}_n: F_n(t_0) \xrightarrow[p = 1]{} F_x
        .\]
    \item $F_n(t)$ является асимптотически нормальной оценкой.
%:: NOTE добавить ссылку на билет где доказаны свойства частоты по типу нормальности 
\end{enumerate}
