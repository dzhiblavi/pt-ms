\section{Основные выборочные оценки и их свойства. Выборочное математическое
ожидание. Выборочная дисперсия. Выборочные моменты. Выборочные медиана и
квантили. Выборочные оценки ковариации и коэффициента корреляции.}
%:: NOTE a:dolzhanskii
%:: NOTE check

\textit{Здесь $X$ -- произвольная рассматриваемая случайная величина.}

\subsection{Выборочное среднее / М.О.}

\begin{definition}
  Случайную величину $\overline{X}_n = EY =
  \sum\limits_{i = 1}^{n} X_{(i)} \cdot \frac{1}{n} =
  \frac{1}{n} \sum\limits_{j = 1}^{n} X_j$ называют \textit{выборочным средним}
  некоторой выборки $X_{[n]}$ из генеральной совокупности $X$.
\end{definition}

Выборочное среднее является выборочной точечной оценкой $EX$.

\paragraph{Свойства.}
\begin{itemize}
  \item Выборка является набором одинаково распределенных независимых случайных
  величин, из чего по законму больших чисел:
  \[
    \overline{X}_n \xlongrightarrow[n \to \infty]{p} EX,~
    \text{если}~ \exists EX
  .\]
  Поэтому выборочное среднее является \textit{состоятельной} оценкой $EX$.

  \item Выборочное среднее является \textit{несмещенной} оценкой $EX$:
  \[
    E_x \overline{X}_n = \frac{1}{n} \sum\limits_{i = 1}^{n} E(X_i) = EX,~
    \text{оо св-ву М.О.}
  \]

  \item Если $\exists EX, DX$, то по центральной предельной теореме:
  \[
    Y_n =
    \frac{\overline{X}_n - E_x(\overline{X}_n)}{\sigma_x(\overline{X}_n)} =
    \frac{\overline{X}_n - EX}{\sigma(X)} \sqrt{n}
    \xlongrightarrow[n \to \infty]{F} Y \sim N(0, 1)
  ,\]
  или
  \[
    F_{Y_n}(t) \overset{\mathbb{R}}{\underset{n \to \infty}{\rcon}} F_Y(t)
  .\]
  Из этого следует, что центрированное нормированное выборочное среднее
  сходится по распределению к стандартному нормальному распределению.
  Следовательно, выборочное среднее является \textit{асимптотически нормальной}
  оценкой $EX$.

\end{itemize}



\subsection{Выборочная дисперсия}

\begin{definition}
  Случайную величину $S^2_n = D(X_{[n]}) =
  \sum\limits_{i = 1}^{n} (X_{(i)} - EY)^2 \cdot \frac{1}{n} =
  \frac{1}{n} \sum\limits_{i = 1}^{n}(X_i - \overline{X}_n)^2$ называют
  \textit{выборочной дисперсией} некоторой выборки $X_{[n]}$ из генеральной
  совокупности $X$.
\end{definition}

Выборочная дисперсия является выборочной точечной оценкой $DX$.

\paragraph{Свойства.}
\begin{itemize}
  \item Выборочная дисперсия является \textit{состоятельной} оценкой $DX$:
  \[
    S^2_n \xlongrightarrow[n \to \infty]{P} E(X^2) - (EX)^2 = DX
  .\]

  \item Выборочная дисперсия является \textit{смещенной} оценкой $DX$.\\
  (Ниже, не теряя общности, будем считать $EX = 0$, инвариантность $DX$
  относительно сдвига):
  \begin{align*}
    &E(X^2) = DX,~ E_x(\overline{X}_n) = EX = 0\\
    &\Rightarrow E_X(\overline{X}^2_n) =
    D_X(\overline{X}_n) + (E_x(\overline{X}_n))^2 = \frac{DX}{n}\\
    &\Rightarrow E_X S^2_n =
    E_X(\frac{1}{n} \sum\limits_{i = 1}^{n} X^2_i - \overline{X}^2_i) =
    \frac{1}{n} n E(X^2) - \frac{DX}{n} = DX - \frac{DX}{n} =
    \frac{n - 1}{n} DX
  .\end{align*}

  \item Выборочная дисперсия является \textit{асимптотически нормальной}
  оценкой $DX$ -- \textit{без доказательства}.
\end{itemize}



\subsection{Несмещенная выборочная дисперсия}
\begin{definition}
  Чаще вместо $S^2_n$ используют \textit{несмещенную (исправленную) оценку
  дисперсии}:
  \[
    \sigma^2_n = \frac{n}{n - 1} S^2_n =
    \frac{1}{n - 1} \sum\limits_{i = 1}^{n} (X_i - \overline{X}_n)
  ,\]
\end{definition}

Несмещенная выборочная дисперсия является выборочной точечной оценкой $DX$.

\paragraph{Свойства.}
\begin{itemize}
  \item \textit{Cостоятельность} следует из состоятельности $S^2_n$:
  \[
    \sigma^2_n \xlongrightarrow[n \to \infty]{p = 1} DX
  .\]
  \item \textit{Несмещенность} очевидна из доказательства смещенности
  выборочного среднего.

  \item Несмещенная оценка дисперсии является \textit{асимптотически нормальной}
  оценкой $DX$ -- \textit{без доказательства}.
\end{itemize}



\subsection{Выборочные моменты}

\subsubsection{Выборочные начальные моменты}

\begin{definition}
  \textit{Выборочным начальным моментом порядка $k$} называется статистика:
  \[
    m_{n, k} = \frac{1}{n} \sum\limits_{i = 1}^{n} X^k_i,~ k = 1, 2, \ldots
  .\]
\end{definition}

Эти выборочные характеристики можно считать выборочным средним для случайной
величины $Z = X^k$:
\[
  m_{n, k} = \overline{Z}_k
.\]

Следовательно, если $\exists E(X^k)$, то $m_{n, k}$ является
\textit{состоятельной} и \textit{несмещенной} оценкой $E(X^k)$.\\

Если существует $E(X^{2k})$, то $m_{n, k}$ является \textit{асимптотически
нормальной} оценкой $E(X^k)$ с асимптотической дисперсией
$\Delta^2 = E_x(Z^2) - (E_x(Z))^2$.

\subsubsection{Выборочные центральные моменты}

\begin{definition}
  \textit{Выборочным центральным моментом порядка $k$} называется статистика:
  \[
    \mu_{n, k} = \frac{1}{n} \sum\limits_{i = 1}{n} (X_i - \overline{X}_n)^k
  .\]
\end{definition}

Данные статистики являются \textit{состоятельными}, \textit{смещенными} оценками
соответствующих центральных моментов генеральной совокупности.



\subsection{Выборочная медиана}

\begin{definition}
  \textit{Медианой $t_0$} случайной величины $X$ ($med(x)$) называют такое значение
  аргумента функции распределения $F_x(t)$, что для него выполняются
  неравенства:
  \[
    \begin{cases}
      P(X \geqslant t_0) \geqslant \frac{1}{2}\\
      P(X \leqslant t_0) \geqslant \frac{1}{2}
    \end{cases}
  .\]
  Если $F_x(t) \in C(\mathbb{R})$, то $F_x(t_0) = \frac{1}{2}$.
\end{definition}

\begin{definition}
  Пусть $X_{(1)} \leqslant X_{(2)} \leqslant \ldots \leqslant X_{(n)}$ --
  упорядоченная выборка (вариационный ряд), тогда \textit{выборочной vедианой
  $med_n$} называется следующая случайная величина:
  \[
    med_n = \begin{cases}
      X_{(k)} = X_{\frac{n - 1}{2}},~ \text{при}~ n = 2k - 1\\
      \frac{X_{(k)} + X_{(k + 1)}}{2},~ \text{при}~ n = 2k
    \end{cases}
  .\]
\end{definition}

\paragraph{Свойства.}
Пусть генеральная совокупность является непрерывной случайной величиной
и $T = \{t: 0 < F_X(t) < 1\}$. Если $f_X(t)$ непрерывна и положительна при
$t \in T$, то плотность распределения случайной величины $Y - f_Y(t)$, где
\[
  Y = 2 \sqrt{n} f_X(t_0) (med_n - t_0),~ t_0 = med(X)
\]
при $n \to \infty$ стремится к $f_{N(0, 1)}(t) = \frac{1}{\sqrt{2 \pi}}
\exp{-\frac{t^2}{2}}$, а
\[
  P(a < Y < b) \xlongrightarrow[n \to \infty]{} \frac{1}{\sqrt{2 \pi}}
  \int\limits_{a}^{b} e^{-\frac{x^2}{2}} dx
.\]
\begin{itemize}
  \item Следовательно, выборочная медиана является \textit{состоятельной}
  оценкой $med(X)$.
  \item Также видно, что выборочная медиана является \textit{асимптотически
  нормальной} оценкой $med(x)$ с асимптотической дисперсией
  $\Delta^2 = \frac{1}{r f^2_X(t_0)}$.
  \item Выборочная медиана является \textit{$\sqrt{n}$-несмещенной} оценкой
  $med(X)$. То есть:
  \[
    \sqrt{n} b_{n, \theta} (med_n) = \sqrt{n} (E_x(med_n) - med(X))
    \xrightarrow[n \to \infty]{} 0
  .\]
\end{itemize}



\subsection{Выборочная ковариация и корреляция}

Выборочная ковариация и корелляция используются при решении вопроса о наличии
зависимости между случайными величинами $X$ и $Y$.\\

В этом случае рассматривается выборка из случайного вектора $(X, Y)$. Здесь
пары $\{X_i, Y_i\}_i$ независимы и одинаково распределены. Если случайные
величины $X$ и $Y$ не являются линейно зависимыми ($r(X, Y) \neq 1$), то для
последовательности $\{X_i, Y_i\}_i$ справедливо утверждение аналогичное
центральной предельной теореме.

\subsubsection{Выборочная ковариация}

\begin{definition}
  \textit{Выборочной ковариацией} называется статистика:
  \[
    K_n = K_n(X, Y) = \frac{1}{n} \sum\limits_{i = 1}^{n}
    ((X_i - \overline{X}_n) (Y_i - \overline{Y}_n)) =
    \frac{1}{n} \sum\limits_{i = 1}^{n} X_i Y_i - \overline{X}_n \overline{Y}_n
  .\]
\end{definition}

\paragraph{Свойства.}
\begin{itemize}
  \item \textit{Состоятельная} оценка.
  $X$ и $Y$.
  \item \textit{Смещенная} оценка. Аналогично выборочной дисперсии, можно
  показать:
  \[
    E_x(K_n) = \frac{n - 1}{n} K(X, Y)
  .\]
  \item \textit{Асимптотически нормальная} оценка.
\end{itemize}

\begin{definition}
  В приложениях обычно рассматривают \textit{несмещенную оценку ковариации}:
  \[
    \widetilde{K}_n = \frac{n}{n - 1} K_n =
     \frac{1}{n - 1} \sum\limits_{i = 1}^{n}
     ((X_i - \overline{X}_n) (Y_i - \overline{Y}_n))
  .\]
\end{definition}

\subsection{Выборочная корреляция}

\begin{definition}
  \textit{Выборочной корреляцией} $X$ и $Y$ называется статистика:
  \[
    r_n = r_n(X, Y) = \frac{K_n}{S_{n, X} S_{n, Y}} =
    \frac{\widetilde{K}_n}{\sigma_{n, X} \sigma_{n, Y}}
  .\]
\end{definition}

\begin{remark}
  В определении выше предполагается существование всех необходимых моментов:
  $EX, EY, K(X, Y), \ldots$
\end{remark}

\paragraph{Свойства.}
\begin{itemize}
  \item \textit{Состоятельная} оценка.
  \item \textit{Несмещенная} оценка.
  \item \textit{Асимптотически нормальная} оценка.
\end{itemize}
